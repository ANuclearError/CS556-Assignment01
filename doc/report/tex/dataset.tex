\documentclass[dataset.tex]{subfiles}
\begin{document}
\section{The Dataset} % (fold)
\label{sec:dataset}
The dataset being used is data taken from the top 100 most popular games on the
Steam platform during the fortnight period starting the 6th of February 2017.
This is determined by the number of users who have played the game during this
two week period. The data was obtained from the SteamSpy API, containing various
fields for the 100 games. While \fullref{table:fields} \footnote{All tables provides an overview of the score spread
and figures can be found in \fullref{sec:figures_tables}} includes all fields provides an overview of the score spread
returned from the API, Section \fullref{sec:preparation}outlines how only certain provides an overview of the score spread
fields were used as part of this study, due to many of the fields covering
similar areas.

\subsection{Data Fields} % (fold)
\label{sub:data_fields}
The data has over 200 fields included, with the majority being based on the user
tag feature Steam offers, which presented a challenge in regards as to their use
in the project. The main fields of the dataset are:

It must be noted that for the Owners, Players Forever and Players 2 Weeks field
all include an additional field that denotes a variance on the figure within
said field, due to being estimations rather than the actual figure. For example,
the game Dota 2 has \(93\,965\,255\) owners with a variance of \(243\,479\),
meaning that the actual number of owners will lie somewhere in the range of
\(93\,965\,255 \pm 243\,479 \) owners. The same principle applies for the other
two fields.
% subsection data_fields (end)

\subsection{Score Rank} % (fold)
\label{sub:score_rank}
Score rank is the value that is of most interest, being the value that the model
will be trying to predict. The score rank is a value from 1 to 100, and is the
percent of games that this game has a higher score than. A score rank in the 90s
indicates a highly acclaimed title, whereas a rank in the 10s means that the
game is highly unpopular amongst parts of the consumer base. 

\fullref{fig:score_rank} provides a basic overview of how the dataset scored.
The scores have been gathered into 10 groups (1-10, 11-20 and so on). A cursory
glance shows that the dataset is skewed towards the higher end, with 53\% of
games having a ScoreRank of 71 or above. While there is still a wide spread of
scores, with 5 being the lowest and 99 being the average. A median of 75 and
mean of 66.19 does show that the vast majority of games do have high ScoreRanks.
This is a natural assumption to make, since these are the most played games on
Steam, it is natural that they will have higher ScoreRanks. The presence of
lower ranked games in this dataset does show that factors beyond their playtime
are in place.
% subsection score_rank (end)

\subsection{Price} % (fold)
\label{sub:price}
The price of a game presents some issues. Since the dataset only provides a
snapshot of the current price, rather than historical data. It can be difficult
to use the price due to the nature of a store having sales. As much, it must be
assumed that these games have experienced similar lifecycles when it comes to
sales. In addition, the number of free games does pose another challenge, as it
can be argued that they should be studied separately due to differences in
business model.

As \fillref{fig:price} shows, the prices of games range from \$0 to \$59.99,
with the median working out as \$19.99, which is the typical price of an older
major AAA title or a recent premium independent title. Of the 28 games under
\$5, 22 of them are free. The mean price is \$21.28, a result of the majority of
games being on the cheaper side of the histogram.
% subsection price (end)

\subsection{ScoreRank and Price} % (fold)
\label{sub:scorerank_and_price}
One of the first points of interest was looking for a connection between price
and score rank. By plotting all games based on their price and score rank, it
could be determined whether or not there was any correlation.

\fullref{fig:score_price} is divided into four sections based on whether price
provides an overview of the score spread or score rank is low or high. The
bottom-right corner is the most populous, with 38 games not including the 6
games that can be seen on the inner edges of the quarter. This indicates that
cheaper games are more likely to have high score ranks than low score ranks.
Expensive games meanwhile, are more even, suggesting that the market is more
critical of games with a higher price point (though including \$30 games does
show a slight edge for positive receptions).

It is apparent that more than just price will need to considered when estimating
score rank.
% subsection scorerank_and_price (end)

\subsection{Publishers} % (fold)
\label{sub:publishers}
Whoever publishes a game can have an impact on the reviews, due to the market
having prejudices for or against that publisher. As such, it is useful to see
which publishers get higher reviews. Naturally, this is best looked at with
regards to publishers with multiple games, the rest have been grouped together
as `other').

\fullref{fig:publishers} shows that Klei Entertainment and Valve are the
provides an overview of the score spread publishers with the highest average
score ranks, indicating that a game published by these companies is likely to be
of high quality to the markets' eyes. Meanwhile, it is clear that Daybreak and
Ubisoft published games are less popular, getting lower reviews. It is likely
that any model for predicting scores will depend on the publisher as part of it.

\subsection{Tags} % (fold)
\label{sub:tags}
Users are able to add tags to games on Steam, which are used to indicate that
the game has certain features or settings. The dataset provides data on 201
tags for all games. Each game data includes the number of users who have tagged
the game with that tag. For example, DOOM has been tagged as `Action' by over
1000 users. 

In order to provide a summary, it was necessary to add additional data about the
tags. To achieve this, a data frame was created providing each tag with the
number of games with that tag and the average score of games with that tag.
A scatter plot was produced between the number of games tagged and the average
score of these games (see \fullref{fig:tab_score}), which was made to determine
if tags could provide a hint as to how a game is scored, and whether or not the
number of games tagged influences this.

It was seen that as the number of games tagged increases, the average score of
this tag goes nearer to the middle, while the tags that appear less often in
games have a wide range. This suggests that finding any link between tags and
scores is not likely to be found. Tags that are not used enough will not be
usable in any practical model, while the tags used very often will converge into
the middle scores.

It was then thought that being restrictive on what games have a tag could change
the results. Consider two games with \(1\,000\,000\) owners and \(10\,000\)
owners, that were each tagged with `Turn-based strategy' \(1\,000'\) times. The
game with few owners should have more ownership of the tag than the game with
more owners, since a higher percentage of owners used the tag. Thus, the data
was modified to convert the tag columns from absolute users, to relative users.
In addition, a cut-off was also included to discard games where the percentage
of users who used the tag was below a specified amount.

Unfortunately, the figure produced (\fillref{fig:tab_scoe_rel}), shows that
little difference is made, with the scatter plot being very similar to the
original. The conclusion that must be drawn from this is that tags are not a
reliable method to predict users.
% subsection tags (end)
% section dataset (end)
\end{document}