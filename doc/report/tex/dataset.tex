\documentclass[dataset.tex]{subfiles}
\begin{document}
\section{The Dataset} % (fold)
\label{sec:dataset}
The dataset being used is data taken from the top 100 most popular games on the
Steam platform during the fortnight period starting the 6th of February 2017.
This is determined by the number of users who have played the game during this
two week period. The data was obtained from the SteamSpy API, containing various
fields for the 100 games. While \fullref{table:fields} \footnote{All tables
provides an overview of the score spread and figures can be found in
\fullref{sec:figures_tables}} includes all fields provides an overview of the
score spread returned from the API, Section \fullref{sec:preparation}outlines
how only certain provides an overview of the score spread fields were used as
part of this study, due to many of the fields covering similar areas.

\subsection{Data Fields} % (fold)
\label{sub:data_fields}
The data has over 200 fields included, with the majority being based on the user
tag feature Steam offers, which presented a challenge in regards as to their use
in the project. The main fields of the dataset are:

It must be noted that for the Owners, Players Forever and Players 2 Weeks field
all include an additional field that denotes a variance on the figure within
said field, due to being estimations rather than the actual figure. For example,
the game Dota 2 has \(93\,965\,255\) owners with a variance of \(243\,479\),
meaning that the actual number of owners will lie somewhere in the range of
\(93\,965\,255 \pm 243\,479 \) owners. The same principle applies for the other
two fields.
% subsection data_fields (end)

\subsection{Score Rank} % (fold)
\label{sub:score_rank}
Score rank is the value that is of most interest, being the value that the model
will be trying to predict. The score rank is a value from 1 to 100, and is the
percent of games that this game has a higher score than. A score rank in the 90s
indicates a highly acclaimed title, whereas a rank in the 10s means that the
game is highly unpopular amongst parts of the consumer base. 

\fullref{fig:score_rank} provides a basic overview of how the dataset scored.
The scores have been gathered into 10 groups (1-10, 11-20 and so on). A cursory
glance shows that the dataset is skewed towards the higher end, with 53\% of
games having a ScoreRank of 71 or above. While there is still a wide spread of
scores, with 5 being the lowest and 99 being the average. A median of 75 and
mean of 66.19 does show that the vast majority of games do have high ScoreRanks.
This is a natural assumption to make, since these are the most played games on
Steam, it is natural that they will have higher ScoreRanks. The presence of
lower ranked games in this dataset does show that factors beyond their playtime
are in place.
% subsection score_rank (end)

\subsection{Price} % (fold)
\label{sub:price}
The price of a game presents some issues. Since the dataset only provides a
snapshot of the current price, rather than historical data. It can be difficult
to use the price due to the nature of a store having sales. As much, it must be
assumed that these games have experienced similar lifecycles when it comes to
sales. In addition, the number of free games does pose another challenge, as it
can be argued that they should be studied separately due to differences in
business model.

As \fullref{fig:price} shows, the prices of games range from \$0 to \$59.99,
with the median working out as \$19.99, which is the typical price of an older
major AAA title or a recent premium independent title. Of the 28 games under
\$5, 22 of them are free. The mean price is \$21.28, a result of the majority of
games being on the cheaper side of the histogram.
% subsection price (end)

\subsection{ScoreRank \& Price} % (fold)
\label{sub:scorerank_price}
One of the first points of interest was looking for a connection between price
and score rank. By plotting all games based on their price and score rank, it
could be determined whether or not there was any correlation.

\fullref{fig:score_price} is divided into four sections based on whether price
provides an overview of the score spread or score rank is low or high. The
bottom-right corner is the most populous, with 38 games not including the 6
games that can be seen on the inner edges of the quarter. This indicates that
cheaper games are more likely to have high score ranks than low score ranks.
Expensive games meanwhile, are more even, suggesting that the market is more
critical of games with a higher price point (though including \$30 games does
show a slight edge for positive receptions).

It is apparent that more than just price will need to considered when estimating
score rank.
% subsection scorerank_price (end)

\subsection{Publishers} % (fold)
\label{sub:publishers}
Whoever publishes a game can have an impact on the reviews, due to the market
having prejudices for or against that publisher. As such, it is useful to see
which publishers get higher reviews. Naturally, this is best looked at with
regards to publishers with multiple games, the rest have been grouped together
as `other').

\fullref{fig:publishers} shows that Klei Entertainment and Valve are the
provides an overview of the score spread publishers with the highest average
score ranks, indicating that a game published by these companies is likely to be
of high quality to the markets' eyes. Meanwhile, it is clear that Daybreak and
Ubisoft published games are less popular, getting lower reviews. It is likely
that any model for predicting scores will depend on the publisher as part of it.

\subsection{Tags} % (fold)
\label{sub:tags}
Users are able to add tags to games on Steam, which are used to indicate that
the game has certain features or settings. The dataset provides data on 201
tags for all games. Each game data includes the number of users who have tagged
the game with that tag. For example, DOOM has been tagged as `Action' by over
1000 users. 

In order to provide a summary, it was necessary to add additional data about the
tags. To achieve this, a data frame was created providing each tag with the
number of games with that tag and the average score of games with that tag.
A scatter plot was produced between the number of games tagged and the average
score of these games (see \fullref{fig:tag_score}), which was made to determine
if tags could provide a hint as to how a game is scored, and whether or not the
number of games tagged influences this.

It was seen that as the number of games tagged increases, the average score of
this tag goes nearer to the middle, while the tags that appear less often in
games have a wide range. This suggests that finding any link between tags and
scores is not likely to be found. Tags that are not used enough will not be
usable in any practical model, while the tags used very often will converge into
the middle scores.

It was then thought that being restrictive on what games have a tag could change
the results. Consider two games with \(1\,000\,000\) owners and \(10\,000\)
owners, that were each tagged with `Turn-based strategy' \(1\,000'\) times. The
game with few owners should have more ownership of the tag than the game with
more owners, since a higher percentage of owners used the tag. Thus, the data
was modified to convert the tag columns from absolute users, to relative users.
In addition, a cut-off was also included to discard games where the percentage
of users who used the tag was below a specified amount.

Unfortunately, the figure produced in \fullref{fig:tag_score_rel}, shows that
little difference is made, with the scatter plot being very similar to the
original. The conclusion that must be drawn from this is that tags are not a
reliable method to predict users.
% subsection tags (end)

\subsection{Owners \& Players} % (fold)
\label{sub:owners_players}
There are three fields used to denote ownership of a game: owners, players
forever and players 2 weeks. Owners denotes the number of users who have
ownership of the title, while players forever indicates how many have played the
game who own it. Players 2 weeks meanwhile, only includes those who played the
game in the two week period.  Free games result in a more artificial inflation
of the first two fields due to including players who may have only played the
game for a very brief session due to it being free. It would be ideal of score
rank was influenced by recent players more-so than the other fields, since it
would avoid this particular issue.

\subsubsection{Owners} % (fold)
\label{ssub:owners}
The number of owners for a game ranges from \(300\,000\) to \(95\,000\,000\),
which is a monumental range to deal with. The mean is 57 million users while the
median is 29 million, which further indicates this vast spread, mainly caused by
the free games. \fullref{fig:owners} shows that games above 12.5 million owners
are more likely to be positively reviewed than negative, but below that it is
not possible to rely on finding a relation between owners and score ranks.
% subsubsection owners (end)

\subsubsection{Players Forever} % (fold)
\label{ssub:players_forever}
Players Forever is very similar to that of owners, and could be expressed as a
percentage rather than a raw value. As such, it would be natural to assume that
the summary and visualisation to be similar to that of owners. Inevitably, 
\fullref{fig:players_f} does indeed show that the relation is the same, meaning
that there is no real difference between using either of these fields to predict
score rank.
% subsubsection players_forever (end)

\subsubsection{Players Two Weeks} % (fold)
\label{ssub:players_two_weeks}
Since this field is more focused on recent activity, it is more likely to
produce different results than players forever did. This field ranges from
one hundred thousand to 9.5 million users. With a median of 2.4 and average of
5.3 million. This spread is a scaled down version of the previous two, but still
exists. However, \fullref{fig:players_t} shows that finding a relation between
these two values is highly confusing. The attempted smoothing shows such a wide
area for error that this particular field feels very useless in estimating
score rank.
% subsubsection players_two_weeks (end)
% subsection owners_players (end)

\subsection{Playtime} % (fold)
\label{sub:playtime}
Data is collected on how many minutes users have played the game, both since the
game's release and in a fortnight period as well. Both the average and median of
these two prices of data are stored for each game. This means that they can be
used to see what patterns emerge from these figures.

\subsubsection{Average Forever} % (fold)
\label{ssub:average_forever}
Average playtime forever shows the average amount of time a user has played
this game. Of the 100 games in the dataset, the smallest is 134 minutes while
the highest average is \(21\,480\) minutes. The average of these averages is
\(3\,260\) minutes while the median is \(3\,885\) minutes.

\fullref{fig:avg_f} shows how this relates to score. Those games which are
closer to \(3\,000\) minutes played on average appear to get a bump in score,
while games not played as often or much more often do have slower scores.
However, the decrease for increased averages is slight, and is mostly caused by
a single outlier, suggesting that a too high average doesn't really cause a
decrease in score rank.
% subsubsection average_forever (end)

\subsubsection{Average 2 Weeks} % (fold)
\label{ssub:average_2_weeks}
This data is the same as the previous column, except focused on the past two
weeks. Over the fortnight period the data snapshots, the games were played for
between 94 and \(2\,000\) with an average of 563 minutes and median of 503.
This means that the most popular games on Steam are played for 10 hours by users
over a two week period.

\fullref{fig:avg_t} shows that generally, as the average increases, the score
will decrease, although this is again a slight decrease exaggerated by the
outlier. Again, the margin for error is rather wide, so this does not provide
extensive data.
% subsubsection average_2_weeks (end)

\subsubsection{Median Playtime} % (fold)
\label{ssub:median_playtime}
The median playtimes shows similar patterns to those of the average playtimes.
\fullref{fig:med_f} and \fullref{fig:med_t} show how they relate to the score
rank. It is clear that there is not one of these figures that provides a strong
estimation of score, and thus would either need to be all included, or all
ignored.
% subsubsection median_playtime (end)
% subsection playtime (end)
% section dataset (end)
\end{document}