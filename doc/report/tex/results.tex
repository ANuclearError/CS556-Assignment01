\documentclass[results.tex]{subfiles}
\begin{document}
\section{Results Analysis} % (fold)
\label{sec:results}
Regrettably, there is little substantial information that can be gained from
the attempts at modelling the data from SteamSpy. The clusters obtained were
focused heavily on genre from the tag columns, while the linear regression
proved to be of little help as well. It may be the case that further
experimenting with columns to focus on during linear regression may have
produced a more accurate model.

The score rank is based on reviews users leave on the Steam service. To be able
to predict this, it must be assumed that the users will review in a rational,
predictable manner. There are many outside influences on how users review a game
that cannot be modelled or stored in a dataset. For example, the review system
has been subject to being hijacked for comedy rather than evaluation of the
product itself, and there is the impact of a vocal minority, where those with
negative feelings are more likely to advertise their complaints. For example,
there was a recent incident where the game Football Manager 2017 became heavily
flooded with negative reviews from the Chinese market due to a lack of official
translation \cite{football_manager}. As a result, the score rank of the game
became heavily dominated by something the dataset cannot take into account, this
makes modelling the score rank very difficult.

The initial visualisations showed that publishers were likely to be a strong
indicator of score ranks, but attempts to model this could not progress due to
issues with handling the categorical nature of the variable. It is likely that
another attempt with focus on this area would provide more encouraging results.
As such, it would be recommended that any further study with this dataset would
aim to focus on this particular aspect.

Overall, the results obtained from this process have been disappointing. At this
point, it is unlikely that a model can be found to predict review scores. If
such a model does exist, and was merely not found, it is most likely to be
obtained from tags and publishers rather than the numerical data. Tags, like
reviews, are controlled by the users themselves, while the users will have
established thoughts on the publishers based on previous work, meaning that
the influence the publisher has on reviews can be more predictable and thus more
reliably modelled. The task then would be trying to determine which tags are
likely to be found in positively reviewed games, and which would be seen in
negatively reviewed games.
% section results (end)
\end{document}