\documentclass[introduction.tex]{subfiles}
\begin{document}
\section{Introduction} % (fold)
\label{sec:introduction}

User reviews are a common source of information for consumers when buying
products. Websites such as Amazon and Netflix provide users with the ability to
give scores to products that they have bought or consumed. As a result, it would
be ideal for manufacturers or producers to be able to predict the scoring of
their product based on existing data.

Steam is a service from Valve Corporation that acts as a digital store-front for
PC games, being the most popular store in the market. In January 2017, the
service reached 14 million concurrent users \cite{steam_concurrent} and saw over
4000 games released in the year 2016, amounting to almost 40\% of the total
library size as well \cite{steam_games_2016}. As a result, it is presenting
users with games that they would wish to purchase has became a challenge for
Valve, attempting to ensure that users are able to discover the games that they
would be interested in.

To tackle this problem, Valve introduced many features to the store. One such
feature is user reviews. As mentioned earlier, users are able to provide their
own reviews to titles that they have purchased (or downloaded in the case of
free games). Users are able to give games a positive or a negative review, these
reviews are aggregated into an overall percentage score, which falls into
categories such as `Overwhelmingly Positive' or `Mostly Negative'.

As the size of the services shows, there is great interest in third-parties
mining and performing analytics upon Steam data that is not immediately
available to the public. One such website dedicated to this is SteamSpy
\cite{steamspy_home}, which uses a random sampling of public profiles to
generate estimations on owners and players of games, which as became 
well-regarded amongst the community. As part of its data collection, it converts
games' user review scores into \emph{score ranks}, which convert the percentage
score calculated by Steam into a percentile ranking, so a score rank of 75\%
means that the game has a higher user review score than 75\% of games on Steam.

The aim of this project is to explore models for estimating this score rank
based on the data collated by Steamspy. Using the top 100 played games in a
fortnight period, various techniques will be investigated to determine if
factors such as price, developer or the median number of minutes played can
determine an accurate estimation of a game's score rank. This report will
provide summaries and visualisation of the data being used, before exploring the
techniques used and outlining the results obtained from these practises.

% The purpose of this project is to try and estimate this score rank, based on the
% data available from SteamSpy. The top 100 games played over the fortnight
% beginning the 6th of February 2017 was taken from SteamSpy, providing data such
% as the number of owners, players and current price, in addition to various
% tags given by users. It is hoped that this information will allow for accurate
% predictions of score ranks, and various techniques will be used to generate
% models capable of doing so.
% section introduction (end)
\end{document}